\documentclass[a4paper,12pt]{article}
\usepackage[utf8]{inputenc}
\usepackage{graphicx}
\usepackage{multicolumn}
\usepackage{multirow}
\usepackage{booktabs}
\usepackage[indent=0pt,skip=3mm]{parskip}
\usepackage{array}
\usepackage{commath} % for abs ||
\usepackage{listings}             % Include the listings-package
\usepackage{tabs}
% \usepackage{colortbl}
\usepackage{url}
\usepackage{datetime}
\usepackage[ top=5cm, left=1.5cm, right=1.5cm, headheight=4cm,bottom=1cm]{geometry}
\usepackage{lastpage} % include last page numbering
\usepackage{fancyhdr}
\usepackage[table]{xcolor}% ctan.org/pkg/xcolor %FOR COLORS
% \usepackage{frame}
\settimeformat{xxivtime}
\setdefaultdate{\ddmmyyyydate}
\hyphenation{matriz}
\graphicspath{{figures/}{./images/}}
\newcommand{\eq}[1]{$#1$}
\newcommand{\head}[1]{{\bfseries #1}}
\newcommand{\header}[2][\tiny]{{\bfseries #1 #2}}

%%%%%%%%%%%%%%%%%%%%%%%%%%%%%%%%%%%%%%%%%%%%%%%%%%%%%%%%%%%%%%%%%%%%%%
%%%%%%%%%%%%%%%%%%%%%%%%%%%%%%%%%%%%%%%%%
\newsavebox{\mytabularheader}
\newsavebox{\mytabularheadertitle}
\setlength{\extrarowheight}{0.1cm}
%----------------------------------------------------------------------------------
%-----------------------------------------------------------------------------------------------
\sbox{\mytabularheadertitle}{%
  \begin{minipage}{.52\textwidth}
    \begin{center}
        \bfseries \scriptsize  UNIVERSIDAD NACIONAL DE SAN AGUSTIN\\
        FACULTAD DE INGENIERÍA DE PRODUCCIÓN Y SERVICIOS\\
        ESCUELA PROFESIONAL DE INGENIERÍA DE SISTEMA\\[3mm]
    \end{center}
  \end{minipage}
}

\sbox{\mytabularheader}{%
    \begin{minipage}{\textwidth}
        \centering
        \begin{tabular}{cp{8cm}c}
            \includegraphics[scale=0.3]{epis_logo.png} & 
            \usebox{\mytabularheadertitle} &
            \includegraphics[scale=0.05]{abet_logo.png} \\
            % \hline
            \multicolumn{3}{c}{Formato: Guía de Práctica de Laboratorio / Talleres / Centros de Simulación}\\
             &\multicolumn{1}{c}{Aprobación:  2022/03/01 Código: GUIA-PRLE-001} &  \\
        \end{tabular}
    \end{minipage}
}
%---------------------------------------
\renewcommand{\headrulewidth}{0pt}
\fancypagestyle{plain}{%
  \fancyhf{}%
  \fancyhf[ch]{\usebox{\mytabularheader}}
}
%--------------------------------------
\pagestyle{plain}
%%%%%%%%%%%%%%%%%%%%%%%%%%%%%%%%%%%%%%%%%%%%%%%%%%%%%%%%%%%%%%%%%%%%%%%%%%%%%%%%%%%%%%%%%%
\definecolor{blackRed}{cmyk}{0,81,76,31}

\begin{document}    
\lstset{language=Python,frame=single, firstnumber=1,basicstyle=\footnotesize,
numbers=left,showspaces=false,showstringspaces=false}   
    \begin{table}[t]
        \centering
        \begin{tabular}{|p{2.3cm}<{:}|m{1.7cm}|m{2.4cm}|m{2cm}|m{3cm}|m{0.6cm}|}
            \multicolumn{6}{c}{\cellcolor{red}{\leavevmode\color{white}\header{INFORMACIÓN BÁSICA}}}\\
            \hline
            \header{ASIGNATURA} & \multicolumn{5}{c}{\header[\footnotesize]{Física Computacional.}}\\
            \hline
            \header{\mbox{TÍTULO DE LA} PRÁCTICA} & \multicolumn{5}{c}{\header[\footnotesize]{Práctica de Ecuación diferencial de Laplace.}}\\
            \hline
            \header{\mbox{NÚMERO DE} PRÁCTICA} & {\header[\footnotesize]{05}} & \header{AÑO LECTIVO:} & {\header[\footnotesize]{2022-A}} & \header{NRO. SEMESTRE:} & \header[\footnotesize]{XI}\\
            \hline
            \header{\mbox{FECHA DE} \mbox{PRESENTACIÓN}} & \header{\today} & \header{HORA DE \mbox{PRESENTACIÓN:}} & \multicolumn{3}{c}{\header[\footnotesize]{\currenttime}}\\
            \hline
            \multicolumn{4}{l}{\header[\footnotesize]{Integrante(s): Alván Ventura Edsel Yael}} & \header{NOTA} & \\
            \hline
            \multicolumn{6}{l}{\header[\footnotesize]{DOCENTE(s):} \header[\footnotesize]{Danny Giancarlo Apaza Veliz.}} \\  
            \bottomrule
        \end{tabular}
    \end{table}
    \title{Práctica 6\\Ecuación de calor\\Física Computacional}
    \date{\vspace{-5ex}}
    \maketitle
    \begin{center}
        Escrito por\\
        Alván Ventura, Edsel Yael\\ \texttt{ealvan@unsa.edu.pe}
        \\[3mm]
        Profesor\\Apaza Veliz, Danny Giancarlo\\ \texttt{dapazav@unsa.edu.pe}\\[3mm]
        \today
    \end{center}
    % \newgeometry{top=2cm}
    \enlargethispage{\baselineskip}
    % \newpage
    \section{Implementación}
    A continuación se muestra una implementación vertical de la matriz que grafica la ecuación de onda.
    
    Para hacer la siguiente implementación se uso:
    \begin{enumerate}
        \item La formula de ecuación de onda en una dimensión.
        \item El lenguaje de programación de Python version 3.
        \item Las libreria \emph{numpy} y la libreria \emph{malplotlib}
        \item Se considero una implementación vertical.
    \end{enumerate}

    \section{Problemas}
    Realice la solución de los siguientes problemas usando la 
    misma lógica mostrada en clase:
    \begin{enumerate}
        \item Use \eq{a = 1}, \eq{b = 1}, \eq{v = 1}, \eq{f(x) = sin(2x)} y \eq{g(x) = 2sin(2x)}.
        \item Use \eq{a = 1}, \eq{b = 1}, \eq{v = 2}, \eq{f(x) = x^2 - x + sin(2x)} y \eq{g(x) = sin(x)}.
        \item Use \eq{a = 1}, \eq{b = 1}, \eq{v = 1}, \eq{f(x) = sin(x)} y \eq{g(x) = 0}.
        \item Use \eq{a = 1}, \eq{b = 1}, \eq{v = 1}, \eq{f(x) = 2x - 1} y \eq{g(x) = 0}.
        \item Use \eq{a = 1}, \eq{b = 1}, \eq{d = 1}, \eq{v = 1},\eq{f(x) = |4x - 1|} y \eq{g(x) = 0}.
    \end{enumerate}
    \subsection{Problema 1}
    Grafique la función de onda, con los siguientes valores:
    \begin{quote}
        \centering
        \eq{a = 1}, \eq{b = 1}, \eq{v = 1}, \eq{f(x) = sin(2x)} y \eq{g(x) = 2sin(2x)}.
    \end{quote}
    \subsection{Problema 2}
    Grafique la función de onda, con los siguientes valores:
    \begin{quote}
        \centering
        \eq{a = 1}, \eq{b = 1}, \eq{v = 2}, \eq{f(x) = x^2 - x + sin(2x)} y \eq{g(x) = sin(x)}.
    \end{quote}
    \subsection{Problema 3}
    Grafique la función de onda, con los siguientes valores:
    \begin{quote}
        \centering
        \eq{a = 1}, \eq{b = 1}, \eq{v = 1}, \eq{f(x) = sin(x)} y \eq{g(x) = 0}.
    \end{quote}
    \subsection{Problema 4}
    Grafique la función de onda, con los siguientes valores:
    \begin{quote}
        \centering
        \eq{a = 1}, \eq{b = 1}, \eq{v = 1}, \eq{f(x) = 2x - 1} y \eq{g(x) = 0}.
    \end{quote}
    \subsection{Problema 5}
    Grafique la función de onda, con los siguientes valores:
    \begin{quote}
        \centering
        \eq{a = 1}, \eq{b = 1}, \eq{d = 1}, \eq{v = 1},\eq{f(x) = |4x - 1|} y \eq{g(x) = 0}.
    \end{quote}
    
\end{document}
